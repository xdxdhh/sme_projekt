\documentclass{article}
\usepackage[czech]{babel}
\usepackage{geometry}
\usepackage{graphicx}
\usepackage{subcaption}
\usepackage{svg}
\usepackage{float}

%\graphicspath{{/home/xdxdhh/skola/3_rocnik/sme/Images/}}

\geometry{
  left=3cm,     % Adjust the left margin
  right=3cm,    % Adjust the right margin
  top=3cm,    % Adjust the top margin
  bottom=2.5cm  % Adjust the bottom margin
}

\setlength{\parindent}{0pt}
\usepackage[varg]{txfonts}

\begin{document}

% Title and author
\title{Semestrální práce do předmětu SME}
\author{Diana Varšíková}
\date{\today}

\maketitle

% Start of the document
\section{Zadání}
Byla provedena studie vlivu podávání jistého léku na pokles jistých rozměrových parametrů lidského oka (CBT1, CBT2, CBT3) a nitroočního tlaku (IOP). 
Všechny hodnoty byly změřeny na začátku léčby (baseline), po dvou a osmi týdnech léčby. 
Pomocí analýzy rozptylu jednoduchého třídění otestujte rozdílnosti hodnot parametrů CBT1, CBT2, CTB3 a IOP naměřených v jednotlivých návštěvách. 
Ověřte podmínky pro použití analýzy rozptylu a výsledky, včetně základních deskriptivnich statistik (průměry, odchylky, odlehlá pozorování atd.), 
uveďte ve formě tabulek a grafů.
   

% Rest of the document
\section{Popis dat}

Měření bylo prováděno na celkem 43 pacientech. Bylo opakováno před začátkem podávání léků, 2 týdny po začátku a 8 týdnů po začátku.
Cílem analýzy je otestovat rozdílnost hodnot parametrů CBT1, CBT2, CBT3 a IOP naměřených při jednotlivých návštěvách lékaře
a tedy ověřit vliv léku na dané parametry. Pro první náhled na dané veličiny lze použít krabicové grafy.

\begin{figure}[H]
  \centering
  \begin{subfigure}{0.45\textwidth}
    \includesvg[width=\textwidth]{Images/violin_CBT1.svg}
    \caption{Krabicový graf CBT1}
  \end{subfigure}
  \begin{subfigure}{0.45\textwidth}
    \includesvg[width=\textwidth]{Images/violin_CBT2.svg}
    \caption{Krabicový graf CBT2}
  \end{subfigure}
\end{figure}

\begin{figure}[H]
  \centering
  \begin{subfigure}{0.45\textwidth}
    \includesvg[width=\textwidth]{Images/violin_CBT3.svg}
    \caption{Krabicový graf CBT3}
  \end{subfigure}
  \begin{subfigure}{0.45\textwidth}
    \includesvg[width=\textwidth]{Images/violin_IOP.svg}
    \caption{Krabicový graf IOP}
  \end{subfigure}
\end{figure}

Po jistém průzkumu se zdá, že veličiny CBT by mohly popisovat spektrální citlivosti lidského oka pro červenou, modrou a zelenou barvu.
Tyto veličiny běžně nabývají hodnot od 0 do 1.
V přiloženém datasetu se nachází dvě hodnoty vyšší než 1 (konkrétně 1.23 a 1.031) u dvou různých pacientů.
Jelikož se zároveň jedná o odlehlé hodnoty od ostatních, tito dva pacienti budou z následné analýzy vynecháni.
Ty odlehlé hodnoty, které jsou ale stále v rozmezí [0,1] budou v datasetu zanechány.
Zbývají tedy data o 41 pacientech.\\

Normální hodnoty nitroočního tlaku se pohybují v rozmezí 10-20 mm Hg, avšak při onemocnění očí, například zákalech, mohou dosahovat i vyšších hodnot.
Při hodnotách nitoočního tlaku nad 30 se jedná již o vážné akutní zákaly.
Pro nitrooční tlak se tedy žádné odlehlé pozorování nevynechalo.

Z krabicových grafů jde vidět, že na začátku mají pacienti daleko vyšší hodnoty nitroočního tlaku než běžná populace.
Dále je patrné že IOP s časem klesá, což by mohlo naznačovat, že pacienti pozitivně reagují na lék.

Pro účely analýzy jsou veličiny označeny předponami W0, W2 a W8.
V následující tabulce jsou shrnuty deskriptivní statistiky po vyřazení odlehlých pozorování.

\begin{table}[ht]
  \footnotesize
  \centering
  \caption{Deskriptivní statistiky po ostranění odlehlých pozorování}
  \begin{tabular}{|ccccccccccccc|}
    \hline
    & \multicolumn{4}{|c|}{Před podáváním} & \multicolumn{4}{c|}{Po 2 týdnech} & \multicolumn{4}{c|}{Po 8 týdnech} \\
    \hline
    & CBT1 & CBT2 & CBT3 & IOP & CBT1 & CBT2 & CBT3 & IOP & CBT1 & CBT2 & CBT3 & IOP \\
    \hline
    průměr & 0.67 & 0.50 & 0.39 & 21.46 & 0.64 & 0.49 & 0.39 & 19.05 & 0.62 & 0.49 & 0.39 & 17.76 \\
    std & 0.12 & 0.08 & 0.09 & 4.30 & 0.13 & 0.10 & 0.06 & 3.67 & 0.15 & 0.11 & 0.07 & 3.77 \\
    min & 0.44 & 0.33 & 0.24 & 13.00 & 0.43 & 0.28 & 0.24 & 14.00 & 0.35 & 0.26 & 0.22 & 11.00 \\
    25\% & 0.56 & 0.44 & 0.34 & 18.00 & 0.54 & 0.42 & 0.35 & 17.00 & 0.50 & 0.41 & 0.35 & 15.00 \\
    50\% & 0.69 & 0.50 & 0.39 & 22.00 & 0.60 & 0.49 & 0.40 & 18.00 & 0.58 & 0.48 & 0.40 & 17.00 \\
    75\% & 0.76 & 0.57 & 0.42 & 25.00 & 0.75 & 0.57 & 0.43 & 22.00 & 0.75 & 0.58 & 0.44 & 19.00 \\
    max & 0.90 & 0.65 & 0.75 & 29.00 & 0.83 & 0.66 & 0.48 & 29.00 & 0.90 & 0.67 & 0.52 & 26.00 \\
    \hline
  \end{tabular}
\end{table}

\newpage

\section{ANOVA - předpoklady}
Pro ověření rozdílnosti parametrů před a po podávání léku použijeme metodu ANOVA, jejíž hypotézou je rovnost středních hodnot při jednotlivých návštěvách.
Alternativou je nerovnost středních hodnot. V případě zamítnutí hypotézy je tedy nutné ještě vyšetřit, které hodnoty zamítnutí způsobily.

Před použitím samotné metody je nutno ověřit předpoklady pro její funkčnost.
Je tedy pro každou ze 4 trojic potřeba ověřit následující :
\bigbreak
1. normalitu dat\\
2. stejné rozptyly dat\\ 
3. nezávislost dat\\

Vzhledem k tomu, že se díváme na vývoj jedné testovací skupiny v čase, tak neplatí předpoklad nezávislosti výběrů
a vhodnější by bylo použít Repeated Measures ANOVA metodu.
Zde bude ale použita pouze jednofaktorová ANOVA a následně správnost nalezených výsledků ověříme ještě párovým t-testem.

K ověření normality dat použijeme pro každou z 12 veličin QQ plot a Lillieforsův test.
Výpočet testu se provede funkcí $\mathbf{lilliefors}$ z Python knihovny $statsmodels$.


\begin{figure}[H]
  \centering
    \includesvg[width=\textwidth]{Images/CBT1_qqplot.svg}
    \caption{QQ plot pro CBT1}
\end{figure}

\begin{table}[H]
  \small
  \centering
  \caption{Výsledky Lillieforsova testu pro CBT1}
  \begin{tabular}{|ccc|}
    \hline
    týden & KS-statistika & p-hodnota\\
    \hline
    W0 & 0.1314 & 0.0718\\
    W2 & 0.1278 & 0.0901\\
    W8 & 0.1364 & 0.0525\\
    \hline
  \end{tabular}
\end{table}

Z QQ plotu lze vidět mírné odchylky od normálního rozdělení, především u první veličiny W0CBT1.
Pro upřesnění použijeme Lillieforsův test. Z něj vidíme, že na hladině 0.05 nezamítáme hypotézu, že data W0CBT1, W0CBT2 a W0CBT3 mají normální rozdělení.

\begin{figure}[H]
  \centering
    \includesvg[width=\textwidth]{Images/CBT2_qqplot.svg}
    \caption{QQ plot pro CBT2}
\end{figure}

\begin{table}[H]
  \small
  \centering
  \caption{Výsledky Lillieforsova testu pro CBT2}
  \begin{tabular}{|ccc|}
    \hline
    týden & KS-statistika & p-hodnota\\
    \hline
    W0 & 0.0899 & 0.5466\\
    W2 & 0.1132 & 0.2116\\
    W8 & 0.0897 & 0.5489\\
    \hline
  \end{tabular}
\end{table}

Pro CBT2 nejsou na QQ plotu zřejmé velké výchylky oproti normálnímu rozdělení, snad kromě pravého chvostu.
Z Lillieforsova testu opět vidíme, že na hladině 0.05 normalitu nezamítáme.

\begin{figure}[H]
  \centering
    \includesvg[width=\textwidth]{Images/CBT3_qqplot.svg}
    \caption{QQ plot pro CBT3}
\end{figure}


\begin{table}[H]
  \small
  \centering
  \caption{Výsledky Lillieforsova testu pro CBT3}
  \begin{tabular}{|ccc|}
    \hline
    týden & KS-statistika & p-hodnota\\
    \hline
    W0 & 0.1482 & 0.0238\\
    W2 & 0.1494 & 0.0223\\
    W8 & 0.1064 & 0.2858\\
    \hline
  \end{tabular}
\end{table}

V QQ plotu pro W0CBT3 vidíme jedno velmi se vychylující pozorování.
Obě p-hodnoty pro W0CBT3 i pro W2CBT3 jsou 0.02, tudíž Lillieforsův test zamítá na hladině 0.05 normalitu pro W0CBT3 i W2CBT3.
Pro CBT3 tedy nebude možné použít ANOVA metodu.

\begin{figure}[H]
  \centering
    \includesvg[width=\textwidth]{Images/IOP_qqplot.svg}
    \caption{QQ plot pro IOP}
\end{figure}

\begin{table}[H]
  \small
  \centering
  \caption{Výsledky Lillieforsova testu pro IOP}
  \begin{tabular}{|ccc|}
    \hline
    týden & KS-statistika & p-hodnota\\
    \hline
    W0 & 0.1310 & 0.0734\\
    W2 & 0.1979 & 0.0009\\
    W8 & 0.1269 & 0.0946\\
    \hline
  \end{tabular}
\end{table}


Pro veličinu W2IOP opět zamítáme hypotézu o normálním rozdělení na hladině 0.05.
Pro veličiny IOP a CBT3 bude tedy nutné použít neparametrickou verzi ANOVA, tedy Kruskalův-Wallisův test.\\


Jako poslední je potřeba otestovat rovnost rozptylů v daných skupinách.
Pro normálně rozdělená data bude pro test rovnosti rozptylů použit Bartlettův test, který se nachází v balíku $scipy.stats$.

\begin{table}[H]
  \small
  \centering
  \caption{Výsledky Bartlettova testu}
  \begin{tabular}{|ccc|}
    \hline
    veličiny & testovací statistika & p-hodnota\\
    \hline
    W0CBT1, W2CBT1, W8CBT1 & 2.3455 & 0.3095\\
    W0CBT2, W2CBT2, W8CBT2 & 3.5366 & 0.1706\\
    W0CBT3, W2CBT3, W8CBT3 & 5.3946 & 0.0673\\
    W0IOP, W2IOP, W8IOP & 1.1761 & 0.5554\\
    \hline
  \end{tabular}
\end{table}

Podle Bartlettova testu nezamítáme pro žádné veličiny na hladině 0.05 hypotézu o rovnosti rozptylů.
Pro data, pro něž jsme zamítli hypotézu o normálním rozdělení bude použit ještě Levene test, který slouží pro porovnání
rozptylů dat, jež nejsou normálně rozdělena. Opět se nachází v balíku $scipy.stats$.

\begin{table}[H]
  \small
  \centering
  \caption{Výsledky Leveneho testu}
  \begin{tabular}{|ccc|}
    \hline
    veličiny & testovací statistika & p-hodnota\\
    \hline
    CBT3 & 0.4516 & 0.6376\\
    IOP & 1.1727 & 0.3130\\
    \hline
  \end{tabular}
\end{table}

Ani tento test nezamítá na hladině 0.05 rovnost rozptylů, tudíž předpoklad rovnosti rozptylů je splněn u všech 4 veličin.

\newpage

\section{ANOVA}

Pro ověření vlivu léku na dané veličiny (CBT1, CBT2, CBT3, IOP) použijeme metodu ANOVA, kterou vždy porovnáme,
jestli má trojice dat (před, 2 týdny po, 8 týdnů po) stejné střední hodnoty.
Hypotézou je rovnost všech tří hodnot, alternativou je alespoň jedna nerovnost.

Pro provedení samotné ANOVA metody lze v Pythonu použít funkce $\mathbf{f\_oneway}$ z balíku $scipy.stats$, která ovšem vrací jenom
hodnotu F statistiky a p-hodnotu.
Pro zisk celé tabulky analýzy rozptylu je nutné požít funkci $\mathbf{ols}$ z $statsmodels.formula.api$ pro tvorbu modelu
a následně na ten model použít funkci $\mathbf{stats.anova\_lm}$ z $statsmodels.api$.

Výsledkem metody je tabulka analýzy rozptylu, která obsahuje počet stupňů volnosti, hodnotu F statistiky, součet čtverců(SS)
a p-hodnoty.

\begin{figure}[H]
  \centering
    \begin{tabular}{|ccccc|}
      \hline
      zdroj & SS & df & F & p-hodnota \\
      \hline
      mezi skupinami & 0.0554 & 2 & 1.603 & 0.205 \\
      v rámci skupin & 2.076 & 120 & - & - \\
      \hline
    \end{tabular}
    \caption{CBT1}
\end{figure}
    

\begin{figure}[H]
  \centering
    \begin{tabular}{|ccccc|}
      \hline
      zdroj & SS & df & F & p-hodnota \\
      \hline
      mezi skupinami & 0.0027 & 2 & 0.1378 & 0.8713 \\
      v rámci skupin & 1.184 & 120 & - & - \\
      \hline
    \end{tabular}
    \caption{CBT2}
\end{figure}

Z tabulek analýzy rozptylu vidíme, že p-hodnoty jsou vyšší než 0.05 a tedy ani v jednom případě nezamítáme hypotézu o shodných středních hodnotnách.

Protože se nám nepodařilo prokázat normální rozdělení veličin CBT3 a IOP, je nutné použít neparametrickou verzi ANOVA metody, a to Kruskalův-Wallisův test. 
Tento test je opět implementován jako $\mathbf{kruskal}$ v knihovně $scipy.stats$.
Hypotéza a alternativa zůstávají stejné jako v předešlých případech.
Výsledky lze vidět v tabulce níže.

\begin{table}[H]
  \small
  \centering
  \caption{Výsledky Kruskal-Wallisova testu}
  \begin{tabular}{|ccc|}
    \hline
    veličiny & testovací statistika & p-hodnota\\
    \hline
    CBT3 & 0.2771 & 0.8706\\
    IOP & 14.9622 & 0.0005\\
    \hline
  \end{tabular}
\end{table}

P-hodnota pro veličinu CBT3 je vyšší než 0.05, tudíž nezamítáme hypotézu shodnosti středních hodnot.
Avšak p-hodnota pro veličinu IOP je pouze 0.0005 a tedy zamítáme hypotézu, že střední hodnoty nitroočního tlaku se s podáváním léku nemění.
Můžeme se znovu podívat na krabicové grafy nitroočního tlaku, tentokrát bez odlehlých pozorování.

\begin{figure}[H]
  \centering
    \includesvg[width=0.5\textwidth]{Images/violin_IOP_after.svg}
    \caption{Krabicový graf nitroočního tlaku po vyřazení odlehlých dat}
\end{figure}

Zdá se, že střední hodnota před podáváním léku je vyšší než po začátku podávání.
Pro podrobnější analýzu rozdílů mezi středními hodnotami použijeme Tukeyeho vícenásobné porovnávání.
Opět lze použít funkci $\mathbf{tukey\_hsd}$ z balíku $scipy.stats$.

\begin{table}[H]
  \small
  \centering 
  \caption{Tukeyeho vícenásobné porovnávání}
  \begin{tabular}{|ccccc|}
    \hline
    porovnání & testovací statistika & p-hodnota & dolní CI & horní CI\\
    \hline
    W0 - W2 & 2.415 & 0.017 & 0.358 & 4.471\\
    W0 - W8 & 3.707 & 0.000 & 1.651 & 5.764\\
    W2 - W0 & -2.415 & 0.017 & -4.471 & -0.358\\
    W2 - W8 & 1.293   & 0.299  &  -0.764  &   3.349\\
    W8 - W0  &   -3.707   &  0.000  &  -5.764 &   -1.651\\
    W8 - W2   &  -1.293  &   0.299   & -3.349   &  0.764\\
    \hline
  \end{tabular}
\end{table}

Z Tukeyeho tabulky vyplývá, že dvojice W0 - W8 i dvojice W0 - W2 mají jiné střední hodnoty.
U dvojice W2 - W8 se naopak nepodařilo prokázat rozdílnost středních hodnot.

Jak již bylo zmíněno, nakonec je nutné ještě výsledky ověřit pomocí t-testu.
Klasický t-test lze použít pokud máme shodné rozptyly a nezávislá data.
Vzhledem k závislosti daných veličin je nutné použít párový t-test na  dvojice (W0IOP, W2IOP) a (W0IOP, W8IOP).
Výsledky vidíme v tabulce níže.
Hypotézou je opět shodnost středních hodnot a alternativou jejich odlišnost.
Pro tento test byla použita funkce $\mathbf{ttest\_rel}$, opět z knihovny $scipy.stats$.

\begin{table}[H]
  \small
  \centering
  \caption{Výsledky párového t-testu testu}
  \begin{tabular}{|ccc|}
    \hline
    dvojice & t-statistika & p-hodnota\\
    \hline
    (W0IOP, W2IOP) & 4.1546 & 0.0.0001\\
    (W0IOP, W8IOP) & 5.8685 & 7.2334e-7\\
    \hline
  \end{tabular}
\end{table}

Párový t-test tedy na hladině 0.05 zamítá shodnost středních hodnot, a tedy potvrzuje výsledky Tukeyeho porovnávání.

\newpage
\section{Závěr}
Cílem bylo ověřit vliv léku na hodnoty CBT1, CBT2, CBT3 a nitrooční tlak.
Již na začátku byl pozorován vyšší nitrooční tlak než je běžný v populaci.
Nejprve jsme na základě interpretace veličin smazali odlehlá pozorování s hodnotami CBT veličin vyššími než 1.
Následně byly ověřeny předpoklady pro použití ANOVA metody, přičemž se nepovedlo prokázat normalitu veličin CBT3 a IOP.\\
Použili jsme tedy ANOVA metodu pro CBT1 a CBT2 a neparametrický Kruskalův-Wallisův test pro CBT3 a IOP.
U veličin CBT1, CBT2 a CBT3 se nepodařilo prokázat odlišnost středních hodnot, a tedy zamítáme vliv používání léku na tyto veličiny.
U veličiny IOP se nám podařila prokázat odlišnost středních hodnot a pomocí Tukeyeho porovnávání jsme zjistili, že střední hodnota nitroočního tlaku před podáváním léku
je odlišná od středních hodnot při podávání.\\
Vzhledem k tomu, že hodnoty nitroočního tlaku u zdravého člověka se pohybují mezi 10 a 20, rozdíl středních hodnot by mohl znamenat, že lék byl úspěšný a snížil nitrooční tlak nemocných pacientů.
Z krabicového grafu vidíme, že hodnoty po 8 týdnech jsou opět nižší než hodnoty po 2 týdnech užívání, ačkoliv se v tomto případě neprokázal rozdíl středních hodnot.

Závěrem tedy je, že lék pomohl ke snížení nitroočního tlaku nemocných pacientů a zároveň neměl žádné vedlejší účinky na spektrální citlivost jejich očí.

\end{document}



